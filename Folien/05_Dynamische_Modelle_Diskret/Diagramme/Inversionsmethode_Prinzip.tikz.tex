\documentclass[tikz,border=5mm]{standalone}
\usepackage{pgfplots}
\pgfplotsset{compat=1.17}

\begin{document}
\begin{tikzpicture}
    \begin{axis}[
        axis lines=middle,
        xlabel=$x$,
        ylabel={$P(X \le x)$},
        xmin=0, xmax=5,
        ymin=0, ymax=1.1,
        xtick=\empty, ytick=\empty,
        no markers,
        domain=0:5,
        samples=100,
        enlargelimits=false,
        axis line style={- latex},
        xlabel style = {at=(current axis.right of origin), anchor=west},
        ylabel style = {at=(current axis.above origin), anchor=south west},
    ]

    % Beispiel-CDF (Exponentialverteilung CDF: F(x) = 1 - e^(-lambda*x))
    % lambda = 1 für Einfachheit
    \addplot[blue, thick, smooth] {1 - exp(-x)};
    \node[blue, above right] at (axis cs: 2.5, 0.9) {$F(x)$};

    % Zufallszahl U auf y-Achse
    \def\randU{0.7} % Beispielwert für U
    \node[left, blue] at (axis cs: 0, \randU) {$U$};
    \draw[dashed, blue] (axis cs: 0, \randU) -- (axis cs: {ln(1/(1-\randU))}, \randU); % Horizontal bis CDF

    % Punkt auf CDF und Vertikal nach x-Achse
    \pgfmathsetmacro{\randX}{ln(1/(1-\randU))}
    \node[below, red] at (axis cs: \randX, 0) {$X$};
    \draw[dashed, red] (axis cs: \randX, \randU) -- (axis cs: \randX, 0); % Vertikal von CDF zu x-Achse

    % Pfeile zur Verdeutlichung des Ablaufs
    \draw[->, very thick, blue] (axis cs: 0.2, \randU) coordinate (u_start) -- (axis cs: \randX - 0.2, \randU) node[above,midway] {$U \to F(x)$};
    \draw[->, very thick, red] (axis cs: \randX, \randU - 0.5) -- (axis cs: \randX, 0.5) node[right,midway] {$F^{-1}(U) \to X$};

    \end{axis}
\end{tikzpicture}
\end{document}
